\documentclass[10pt,journal,twocolumn]{IEEEtran}
\usepackage[noadjust]{cite}
%important package
\usepackage{multirow} 
\usepackage{algpseudocode}
\usepackage{algorithm}
\usepackage{rotating}
\usepackage{kantlipsum} %with the next two command (commath, allowdisplaybreaks) -> allow to break the formulas along the pages
\usepackage{commath}
\allowdisplaybreaks
\usepackage{mathtools}  %with the next command adjust the vertical space between formulas
%\setlength{\jot}{5pt}
%\usepackage{ragged2e}   %if add \justify at each line, that line would be justified. This is used because the abstract of transaction style is not justified  ---- it is not compatible with twocolumn-IEEETrans
%not important
\usepackage{verbatim}
\usepackage{xr-hyper} 
\usepackage{enumitem}
%\usepackage{fullpage} 
%%%%%%%%%%%%%%%%%%%%%%%%%%%%%%%%%%%%%%%%%%%%%%%%%%%%%%%%%%%%%%%%%%%%%%%%%%%%%
% hyperref package may need to be commented for Latex upload
%\usepackage[pdfusetitle, pdfauthor={Michael Shell, My institution}]{hyperref}
%%%%%%%%%%%%%%%%%%%%%%%%%%%%%%%%%%%%%%%%%%%%%%%%%%%%%%%%%%%%%%%%%%%%%%%%%%%%%

%\usepackage{bibentry}
%\nobibliography{references}

\usepackage{epstopdf}
\usepackage{wrapfig}
\usepackage{latexsym}
\usepackage{amssymb}
\usepackage{amsthm}
\usepackage{amsfonts}
\usepackage{amsmath} %[cmex10]
%\usepackage{flushend} %********************* This package has a bug: Do no include it
\usepackage{graphicx}
\usepackage{latexsym}
\usepackage{booktabs}
\usepackage{subcaption} %******************* This package has conflict with sufig and subfigure
%\usepackage{subfigure}
%\usepackage{subfig}
\captionsetup[table]{labelsep=space,justification=centering,font=sc}

\usepackage[T1]{fontenc}

\usepackage{breqn}
\newtheorem{thm}{Property}
\newtheorem{thm1}{Theorem}
\newtheorem{thm3}{Proposition}
\newtheorem{thm5}{Remark}
\newtheorem{thm7}{Lemma}
\algnewcommand\algorithmicinput{\textbf{INPUT:}}
\algnewcommand\INPUT{\item[\algorithmicinput]}
\algnewcommand\algorithmicoutput{\textbf{OUTPUT:}}
\algnewcommand\OUTPUT{\item[\algorithmicoutput]}
%\usepackage[table]{xcolor}
\usepackage[dvipsnames]{xcolor}
\newcommand{\note}[1]{\textcolor{blue}{\textbf{[#1]}}}

% To remove all comments comment the line \commentsontrue 
\newcommand{\commentBy}[3]{\textcolor{#1}{\textbf{#2:} #3}}
%\newcommand{\commentBy}[3]{\ignorespaces}

\newif\ifcommentson
%uncomment the line below to show comments
\commentsontrue

%define comments for different authors hereafter
\newcommand{\plv}[1]{\ifcommentson \commentBy{RoyalPurple}{PLV}{#1} \fi}
\newcommand{\ste}[1]{\ifcommentson \commentBy{blue}{SS}{#1} \fi}
\newcommand{\fc}[1]{\ifcommentson \commentBy{red}{FC}{#1} \fi}
\newcommand{\pc}[1]{\ifcommentson \commentBy{red}{PC}{#1} \fi}

\newif\ifextended
\newif\ifshortver
%%% Show only short version in black
%%%        \shorvetrue 
%%%        %\extendedtrue 

%%% Show only extended version in black
%%%        %\shorvetrue 
%%%        \extendedtrue 

%%% Show short version in blue and extended version in purple
%%%        \shorvetrue 
%%%        \extendedtrue 

\shortvertrue
%\extendedtrue

\newcommand{\extended}[1]{\ifextended \ifshortver \textcolor{purple}{#1} \else \textcolor{black}{#1} \fi  \fi}
\newcommand{\shortver}[1]{\ifshortver \ifextended \textcolor{blue}{#1} \else \textcolor{black}{#1} \fi \fi}

\newcommand{\optional}[1]{\ignorespaces}
% correct bad hyphenation here
\hyphenation{net-works fa-ci-li-ta-tes fa-ci-li-ta-te mo-ni-to-ring par-ti-cu-lar pe-ri-o-di-cal-ly mi-ni-mi-zing va-ria-tions de-li-ve-red per-ri-o-di-cal-ly}

\begin{document}

\bstctlcite{IEEEexample:BSTcontrol}

\title{Segment Routing: A comprehensive survey of research activities, standardization efforts and implementation results}

\author{Pier Luigi Ventre, Stefano Salsano, Marco Polverini, Antonio Cianfrani\\
Ahmed Abdelsalam, Clarence Filsfils, Pablo Camarillo, Francois Clad
\IEEEcompsocitemizethanks{\protect
\IEEEcompsocthanksitem P.L. Ventre is with with the Department of Electronic Engineering at the University of Rome Tor Vergata - Rome, Italy, E-mail: pier.luigi.ventre@uniroma2.it.
\IEEEcompsocthanksitem S. Salsano is with the Department of Electronic Engineering at the University of Rome Tor Vergata and with the Consorzio Nazionale Interuniversitario per le Telecomunicazioni (CNIT) - Rome, Italy, E-mail: stefano.salsano@uniroma2.it.
\IEEEcompsocthanksitem M. Polverini and A. Cianfrani are with Department of Information Engineering, Electronics and Telecommunications at the University of Rome Sapienza - Rome, Italy, E-mail: \{marco.polverini, antonio.cianfrani\}@uniroma1.it.
\IEEEcompsocthanksitem A. Abdelsalam, C. Filsfils, P. Camarillo and F. Clad are with Cisco Systems, E-mail: \{ahabdels, cfilsfil, pcamaril, fclad\}@cisco.com
}
\vspace{2ex}
\textbf{\\Submitted paper under review - v02 - April 2019}
\vspace{-3ex}
}

\markboth{\shortver{Submitted to IEEE Communications Surveys \& Tutorials}}
{P.L. Ventre \MakeLowercase{\textit{et al.}}: Segment Routing: A comprehensive survey of research activities, standardization efforts and implementation results}

\IEEEtitleabstractindextext{
\begin{abstract}
Telco Operators were suffering from lack of flexibility and scalability to realize an increasing variety of cloud based services, meet the demand for more bandwidth and keep the costs low and infrastructures agile. Since 2012 Segment Routing (SR) architecture, also known as Source Packet Routing In Networking (SPRING) in the Internet Engineering Task Force, has been proposed as solution aimed at filling this gap. SR founds its architecture on the source routing and tunneling paradigms, becoming a promising solution to support services like Traffic Engineering and Virtual Private Networks in IP backbones and datacenters. SR architecture has also interesting scalability properties, as it reduces the amount of state information that needs to be configured in the core nodes since devices steer packets over paths using a sequence of instructions (segments) placed in the packet header at the ingress of the network. Two instantations exist of the Segment Routing architecture: it can be implemented using either the MPLS or the IPv6 data plane. Recently, the IPv6 SR architecture (SRv6) has been extended from the simple steering of packets across nodes to a general network programming approach by the Internet draft SRv6 Network Programming making SRv6 architecture appealing to new use cases like Service Function chaining and Network Function Virtualization.

In this paper we present a comprehensive survey on SR technology analyzing research activities, standardization efforts and implementation results. Firstly, we introduce the motivation for Segment Routing and its evolution from its public first in 2012. Then, we provide a common ground and a reference conceptual framework for the survey using a bottom-up approach. Next, we analyze research activities using a categorization approach. We have identified 7 main categories during our analysis of the current state of the art. They are respectively Monitoring, Traffic Engineering, Failure Recovery, Centrally Controlled Architectures, Path Encoding, Network Programming and Miscellaneous. We also look to the path of the standardization efforts, we describe the main 9 standardization efforts and we list other ongoing activities and provide pointers to them. Last but not least, we elaborate on the implementation results and we provide references for tutorials, source code and ready-to-go virtual setups.
\end{abstract}

\begin{IEEEkeywords} 
Segment Routing, MPLS, IPv6, SR-MPLS, SRv6, Source Routing, Monitoring, Traffic Engineering, Failure Recovery, Centrally Controlled Architectures, Path Encoding, Networking Programming, Performance, Linux kernel, VPP, Data Plane, Control Plane, Southbound APIs, Northbound APIs, Software Routers, Hardware Routers, Open Source, Software Defined Networking, SDN, Network Function Virtualization, NFV, Service Function Chaining, SFC, Standards 
\end{IEEEkeywords}
}
\maketitle
\IEEEdisplaynontitleabstractindextext
\IEEEpeerreviewmaketitle

\section{Introduction}
\label{sec:intro}

\IEEEPARstart{S}egment Routing (SR) is based on the loose Source Routing concept. A node can include an ordered list of instructions in the packet headers. These instructions steer the forwarding and the processing of the packet along its path in the network. The single instructions are called \textit{segments}, a sequence of instructions can be referred to as a \textit{segment list} or as an \textit{SR Policy} . Each segment can enforce a topological requirement (e.g. pass through a node or an interface) or a service requirement (e.g. execute an operation on the packet). The term \textit{segment} refers to the fact that a network path towards a destination can be split in segments by adding intermediate way-points. The segment list can be included by the original source of the packet or by an intermediate node. When the segment list is inserted by an intermediate node, it can be removed by another node along the path of the packet, supporting the concept of \textit{tunneling} through an \textit{SR domain} from an \textit{ingress} node to an \textit{egress} node. 

The research and standardization activities on Segment Routing originated in the late 2000's, mainly with the goal of overcoming some scalability issues and limitations \cite{rfc5439} that had been identified in the traffic engineered Multi Protocol Label Switching (MPLS-TE) solutions used for IP backbones. In particular it was observed that MPLS-TE requires explicit state to be maintained at all hops along an MPLS path and this may lead to scalability problems in the control-plane and in the data-plane. Moreover, the per-connection traffic steering model of MPLS-TE does not easily exploit the load balancing offered by Equal Cost MultiPath (ECMP) routing in plain IP networks. On the other hand, Segment Routing can steer traffic flows along traffic engineered paths with no per-flow state in the nodes along the path and exploiting ECMP routing within each segment. 

In the early 2010's, the IETF started the ``Source Packet Routing in Networking'' Working Group (SPRING WG) to deal with Segment Routing. The activity of the SPRING WG has included the identification of Use Cases and Requirements for Segment Routing (for example, \cite{rfc7855}, \cite{rfc8355} and \cite{rfc8354} have become IETF RFCs). Recently, the WG has issued the ``Segment Routing Architecture'' document (RFC 8402 \cite{rfc8402}), while several other documents are still under discussion by the WG, as it will be analyzed later in this paper.

The implementation of the Segment Routing Architecture requires a Data Plane which is able to carry the segment lists in the packet headers and to properly process them. Control Plane operations complement the Data Plane functionality, allowing to allocate segments (i.e. associate a segment identifier to a specific instruction in a node) and to distribute the segment identifiers within an SR domain.

As for the Data Plane, two instantiations of the SR Architecture have been designed and implemented, SR over MPLS (SR-MPLS) and SR over IPv6 (SRv6). With SR-MPLS, no change to the MPLS forwarding plane is needed \cite{id-segment-routing-mpls}. SRv6 is based on a new type of IPv6 routing header called SR Header (SRH) \cite{ietf-6man-segment-routing-header}. Temporally, SR-MPLS has been the first instantiation of the SR Architecture, while the recent interest and developments are focusing on SRv6. In particular, the IPv6 SR dataplane is being extended to support the so-called SRv6 Network Programming Model \cite{id-srv6-network-prog}. According to this model, Segment Routing functions can be combined to achieve an end-to-end (or edge-to-edge) networking objective that can be arbitrarily complex. This is appealing for implementing complex services like Service Function Chaining. SRv6 can be used as an \textit{overlay} tunneling mechanism directly exposed and used by servers (similar to VXLAN tunneling) and as a transport mechanism in network backbones (supporting Traffic Engineering and Resilience services). In this vision, SRv6 can simplify network architectures avoiding the use of different protocol layers. 

As for the SR Control Plane operations, they can be based on a distributed, centralized or hybrid approach. In the distributed approach, the routing protocols are used to signal the allocation of segments and the nodes take independent decisions to associate packets to the segment lists. In the centralized approach, an SR controller allocates the segments, takes the decision on which packets needs to be associated to which segment lists and configures the nodes accordingly. Very often, an hybrid approach which consists in the combination of distributed and centralized approach is used (see for example \cite{ventre2018sdn}). 

The goal of this paper is to provide a comprehensive survey on the Segment Routing technology, including all the achieved results and the ongoing work. We will consider the the research activities (section \ref{sec:research}), standardization efforts (section \ref{sec:standard}), the implementation results and ongoing deployments (section \ref{sec:tools}). In addition, in section \ref{sec:arch} we provide a short introduction to the main concepts of the Segment Routing architecture, considering both the SR-MPLS and the SRv6 dataplanes. In the conclusion section we highlight future research directions and open issues.  

To the best of our knowledge, there is only another survey about Segment Routing, namely \cite{abdullah2018segment}. With respect to \cite{abdullah2018segment}, our survey is more complete in the analysis of the scientific literature, covering \numTotalPapers papers. Moreover, it provides a classification and discussion of standardization activities, and considers the status of implementations and deployments of Segment Routing (the analysis of standardization efforts and the analysis of the status of implementations are both missing in \cite{abdullah2018segment}). As for the tutorial part, we deal with IPv6 dataplane (SRv6), which is currently rising a lot of interest and is missing in \cite{abdullah2018segment}.
\section{The Segment Routing (SR) architecture}
\label{sec:arch}

This section includes a short introduction to the main SR architectural aspects. Our goal is to provide a common ground and a reference conceptual framework for the survey, rather than to cover the details in a tutorial way.

The seminal paper on the Segment Routing Architecture is \cite{filsfils2015segment}. Published in 2014, it provides an overview of the motivations for SR, describes a set of important use cases and illustrates the architecture. The basic concepts proposed in \cite{filsfils2015segment} have been elaborated and refined in the RFC 8402 \cite{rfc8402} which has recently completed its standardization process in the IETF (July 2018). Obviously, RFC 8402 \cite{rfc8402} represents the most important source of information for the SR architecture. The work in \cite{sr-ietf-journal} (published in 2017) provides a short and effective introduction to the Segment Routing architecture, with focus on the MPLS dataplane. The survey paper \cite{abdullah2018segment} has a section about the SR architecture, which tries to give more details related to both the Data Plane and the Control Plane aspects.  

Following the RFC 8402, let us start by discussing the general concepts of SR, which are independent from the specific dataplane (MPLS or IPv6). An \textit{SR policy} is an ordered list of segments (segment list). As shown in Fig.~\ref{fig:sr_operations}, the segment list is added to the packet headers by a \textit{headend} node that \textit{steers} the packets of a flow onto the SR policy. A Segment Routing domain (\textit{SR domain}) is the set of nodes participating in the source-based routing model. The headend node can be the originator of the packet or (as Fig.~\ref{fig:sr_operations}) an intermediate node that performs a classification of the traffic and associates the SR policies to the packets. In other words, hosts \textit{can} be part of an SR domain, but this is not required and depends on the overall scenario in which SR is applied. It is expected that all nodes in an SR domain are managed by the same administrative entity. For example, a Service Provider backbone can constitute an SR domain and the headend node will be the ingress edge router of the backbone (in this case hosts are not part of the SR domain).

A segment is described by a Segment Identifier (\textit{Segment ID} or \textit{SID}). For the SR-MPLS dataplane, a SID is an MPLS label, while for the SR-IPv6 dataplane a SID is an IPv6 address. An SR policy P consisting in steering a packet along three segments with SIDs S1, S2 and S3 can be represented as P=<S1,S2,S3> (see Fig.~\ref{fig:sr_operations}). Three basic operations on SIDs and segment lists have been defined for a generic SR dataplane: PUSH, NEXT and CONTINUE. We assume for simplicity that S1, S2 and S3 represent topological instructions, so that the policy P instructs the packet to cross three nodes in sequence, identified by the SIDs S1, S2 and S3. 
%Note that these operations have been originally defined having the MPLS data plane in mind and then they can be remapped for the IPv6 dataplane.
The PUSH operation consists in the insertion of a segment on top of the segment list, i.e. as the new first segment of the SR policy. In order to build the SR policy P described above, the headend node executes the PUSH operations in this order: PUSH(S3), PUSH(S2), PUSH(S1). In an SR packet, the segment that specifies the instruction to be executed is called \textit{active} segment. In the considered example with the SR policy P, the headend node will send the packet with active segment S1. The NEXT operation is executed by a node that has processed the active segment and considers the next segment of the SR policy to be executed. In our example, the node identified with SID S1 receives the packet and perform the NEXT operation. The next segment is S2, which becomes the active segment and the packet is forwarded toward S2. The NEXT operation also covers the case of the last node of an SR policy, in which the NEXT operation usually results in processing the packet according to regular IP forwarding. The CONTINUE operation is performed by nodes that are in the path between two segments. For example, the intermediate nodes in the path between S1 and S2 perform the CONTINUE operation. The path between S1 and S2 is not prescribed by the SR policy and will be chosen considering the regular IP routing toward node S2 in the SR domain. If there are multiple equal cost paths between nodes S1 and S2 (as in Fig.~\ref{fig:sr_operations}) and the ECMP (Equal Cost MultiPath) mechanism is supported by the IP routing in the SR domain, it can be conveniently exploited by Segment Routing. 

\begin{figure}
    \centering
    \includegraphics[width=0.45\textwidth]{fig/sr-domain.pdf}
    \caption{Example of an SR policy and of SR operations}
    \label{fig:sr_operations}
    \vspace{-3ex}
\end{figure}
%pdf generated by powerpoint, printing on PDF creator 
% settings, advanced settings, papersize postscript custom page setup
% width 95 mm height 240 mm % print quality 1200 dpi % true type download as softfonts
% in powerpoint settings "High quality" from the slides print setting


The segments can be classified into \textit{Global Segments} and \textit{Local Segments}. Global Segments correspond to instructions that are globally valid in an SR domain. Local segments correspond to instructions that are valid within a single node. The typical example of a global segment is an instruction to forward packets towards a given destination IP network or a destination IP node. Considering that an IGP (Interior Gateway Protocol) routing protocol (e.g. OSPF or ISIS) is used in the SR domain, these instructions are called \textit{IGP-prefix segment} and \textit{IGP-node segment} (or simply prefix segment and node segment). All nodes in the SR domain can execute the prefix segment or node segment instructions by considering the path towards the destination network or destination node in their own routing table. The most important example of local segment is the instruction to forward a packet to a node identified as adjacent by the IGP routing protocol. This corresponds to sending the packet on a specific outgoing interface and can be executed only on a specific node. This instruction is called \textit{IGP-Adjacency Segment}. Thanks to the use of IGP-Adjacency segments, it is possible to prove that any path across an SR domain can be expressed by an SR Policy (which can include a combination of global and local segments) \cite{pmsr}. Local segments can also be used to represent service instructions to be executed in a given node. The mapping of global and local segment into Segment Identifiers (SIDs) and the distribution of the SIDs in an SR domain are specific to the two different dataplanes (MPLS and IPv6) and will be discussed in the next subsections.

The \textit{IGP-Anycast Segment} is an IGP-Prefix segment that corresponds to an anycast prefix, i.e. a prefix advertised by a set of routers that can be used for High Availability or Load Balancing purposes. 

The \textit{Binding Segment} is used to associate an SR policy to a SID (called Binding SID or \textit{BSID}). A packet received with the BSID will be steered on the associated SR policy, this means that the packet will be forwarded using the corresponding Segment List. Using the Binding Segment it is possible to separate the processes of packet classification by the enforcing of a specific SR Policy. The SR policy can be changed over time (and can be executed in a different node) with no need to change the classification process. This improves the scalability, resilience and service independence of the solutions based on Segment Routing.

Segment Routing is supported by two different dataplanes: i) MPLS and ii) IPv6. Table~\ref{table-sr-mappings} summarizes the mapping of the SR concepts into the two dataplanes and will be discussed in the next two subsections. 

\begin{table}
\caption{\\Mapping SR concepts into SR-MPLS and SRv6}
\label{table-sr-mappings}
\begin{tabular}{|l|l|l|}
\hline
\textbf{Generic SR}                                          & \textbf{SR-MPLS} & \textbf{SRv6}                                                                                                                               \\ \hline
SR Policy                                                    & Label Stack      & \begin{tabular}[c]{@{}l@{}}Segment List (of IPv6\\ addresses) in the SR Header\end{tabular}                                                 \\ \hline
Active Segment                                               & Topmost Label    & \begin{tabular}[c]{@{}l@{}}IPv6 address indicated\\ by the Segment Left field\end{tabular}                                                  \\ \hline
\begin{tabular}[c]{@{}l@{}}PUSH\\ Operation\end{tabular}     & Label Push       & \begin{tabular}[c]{@{}l@{}}Adding an IPv6 in the Segment\\ List in the SR Header\end{tabular}                                               \\ \hline
\begin{tabular}[c]{@{}l@{}}NEXT\\ Operation\end{tabular}     & Label POP        & \begin{tabular}[c]{@{}l@{}}Decrementing the Segment Left\\ field, copying the active segment\\ in the IPv6 Destination Address\end{tabular} \\ \hline
\begin{tabular}[c]{@{}l@{}}CONTINUE\\ Operation\end{tabular} & Label Swap       & \begin{tabular}[c]{@{}l@{}}Forwarding according to IPv6\\ Destination Address\end{tabular}                                                  \\ \hline
\end{tabular}
\end{table}

\subsection{MPLS dataplane (SR-MPLS)}
\label{sec:mpls-dataplane}

The MPLS dataplane (SR-MPLS) is specified in \cite{id-segment-routing-mpls}. For SR-MPLS, Segment Routing does not require any change to the MPLS forwarding plane. An SR Policy is instantiated through the MPLS Label Stack: the Segment IDs (SIDs) of a Segment List are inserted as MPLS Labels. 
The classical forwarding functions available for MPLS networks allow implementing the SR operations. The PUSH operation corresponds to the Label Push function, i.e. pushing one or more labels on top of an incoming packet and then sending it out of a particular outgoing interface. The NEXT operation corresponds to the Label Pop function, i.e. removing the topmost label. The CONTINUE operation corresponds to Label Swap function, i.e. removing the incoming label and inserting an outgoing one. 
The encapsulation of an IP packet into a SR-MPLS packet is performed at the edge of an SR-MPLS domain if the destination address of the packet matches a pre-configured Forwarding Equivalent Class (FEC) associated with a specific SR Policy.

The mapping of Segments to MPLS Labels (SIDs) is a critical process in the SR-MPLS dataplane. In the general case, different routers in the SR domain could have different available ranges of labels to be used for Segment Routing. Therefore each router can advertise its own available label space to be used for Global Segments called \textit{SRGB - Segment Routing Global Block} (in general, this label space can even be composed of a set of non contiguous blocks). For this reason, in the SR domain the Global Segments are identified by an index, which has to be re-mapped into a label taking into account the node that will process the label. Assuming that the SRGB of a node is a label range starting from 10000, for a Global Segment with index X, the node needs to receive the label 10000+X. As an example, in Fig.~\ref{fig:mpls-dataplane}~A we consider how to implement the SR policy described in Fig.~\ref{fig:sr_operations} using the SR-MPLS dataplane in the general case in which different nodes are using different SRGBs. The SRGBs of the nodes and the segment index associated to the segments S1, S2 and S3 are shown in the gray rectangle. The headend node needs to consider in advance which is the SRGB of the nodes that will perform the NEXT operation the segments, because the label for the next segments needs to be crafted accordingly. In particular, the initial label for segment S2 set by the headend node will be 1002, i.e. the SRGB of node S1 (1000) plus the index for segment S2 (2). Node S1 will have to modify the label to 4002 if the packet is forwarded to node N4 (whose SRGB is 4000) or to label 6002 if the packet is forwarded to node N6 (whose SRGB is 6000). Both nodes N4 and N6 will remap (swap) the label to 2002 when forwarding the packets to S2. The initial label for node S3 set by the headend node is 2003, i.e. the SRGB of node S2 (2000) plus the index for segment S3 (3). This label will reach node S2 unmodified, then it will be properly processed by node S2 that will remap (swap) it considering the SRGB of the next hop in the path towards node S3. This remapping process complicates the operations and the troubleshooting. There are also services (e.g. involving anycast segments) that cannot be realized if different SRGBs are used by different nodes. For this reason, \cite{rfc8402} strongly recommends that an identical range of labels (SRGB) is used in all routers, so that a Global Segment will always be mapped to the same SID (MPLS label) in all nodes. In Fig~\ref{fig:mpls-dataplane}~B we present the mapping of the same SR policy described in Fig.~\ref{fig:sr_operations} under the suggested operating mode in which an identical SRGB is used in all nodes. We observe that the MPLS labels do not need to be remapped, so that the same label consistently identifies the same segment throughout the SR domain.

\begin{figure}
    \centering
    \includegraphics[width=0.48\textwidth]{fig/srgb-mpls-ok.pdf}
    \caption{SR-MPLS dataplane: mapping segments to labels using the SRGB}
    \label{fig:mpls-dataplane}
    \vspace{-3ex}
\end{figure}
%pdf generated by powerpoint, printing on PDF creator 
% settings, advanced settings, papersize postscript custom page setup
% width 195 mm height 200 mm % print quality 1200 dpi % true type download as softfonts
% in powerpoint settings "High quality" from the slides print setting


\subsection{IPv6 dataplane (SRv6)}
\label{sec:ipv6 dataplane}

For the IPv6 dataplane (SRv6), a new type of IPv6 Routing Extension Header, called Segment Routing Header (SRH) has been defined in \cite{ietf-6man-segment-routing-header}. The format of the SRH is shown in Fig.~\ref{fig:sr-header}. The SRH contains the Segment List (SR Policy) as an ordered list of IPv6 addresses: each address in the list is a SID. A dedicated field, referred to as \textit{Segments Left}, is used to maintain the pointer to the active SID of the Segment List. To explain the SRv6 dataplane, we consider three categories of nodes: Source SR nodes, Transit nodes and SR Segment Endpoint nodes. A Source SR node corresponds to the \textit{headend} node discussed above. It can be a host originating an IPv6 packet, or an SR domain ingress router encapsulating a received packet in an outer IPv6 header. In Fig.~\ref{fig:srv6-dataplane} we consider the latter case, the Source SR node is an edge router that encapsulates a packet (which can be IPv6, IPv4 or even a Layer 3 frame) into an outer IPv6 packet and inserts the SR Header (SRH) as a Routing Extension Header in the outer IPv6 header. The encapsulated packet is indicated as Payload in Fig.~\ref{fig:srv6-dataplane}. The Segment List in the SRH is composed of S1, S2 and S3 which are stored in reverse order (the fist SID is S3, the last segment in the SR policy). The Segment Left field is set to 2, so that the active segment is S1, represented in red in the figure. The Source SR node sets the first SID of the SR Policy (S1) as IPv6 destination address of the packet. These operations correspond to a sequence of the PUSH operations described above. The SR Segment Endpoint node receives packets whose IPv6 destination address is a local address of an interface or is locally configured as a segment. The SR Segment Endpoint node inspects the SR header: it detects the new active segment, i.e. the next segment in the Segment List, modifies the IPv6 destination address of the outer IPv6 header and forwards the packet on the basis of the IPv6 forwarding table. These operations correspond to the NEXT operation described above. In Fig.~\ref{fig:srv6-dataplane}, we can see that S1 is the first SR Endpoint node, it decrements the Segment Left fields to 1, making S2 the active segment, and sets S2 as IPv6 Destination Address. A Transit node forwards the packet containing the SR header as a normal IPv6 packet, i.e. on the basis of the (outer) IPv6 destination address, because the IPv6 destination address is not a local address, nor it has been configured as a segment. This behavior corresponds to the CONTINUE operation. In Fig.~\ref{fig:srv6-dataplane}, nodes N4, N5, N6 and N7 are Transit nodes, which perform a regular forwarding of the packet towards the IPv6 Destination Address. Note that in SRv6 the Transit nodes do not need to be SRv6 aware, as every IPv6 router can act as an SRv6 Transit node. 

\begin{figure}
    \centering
    \includegraphics[width=0.8\columnwidth]{fig/sr-header.png}
    \caption{Segment Routing Header}
    \label{fig:sr-header}
%    \vspace{-3ex}
\end{figure}

\begin{figure}
    \centering
    \includegraphics[width=0.48\textwidth]{fig/srv6-example.pdf}
    \caption{SRv6 dataplane operations}
    \label{fig:srv6-dataplane}
    \vspace{-3ex}
\end{figure}
%pdf generated by powerpoint, printing on PDF creator 
% settings, advanced settings, papersize postscript custom page setup
% width 118 mm height 230 mm % print quality 1200 dpi % true type download as softfonts
% in powerpoint settings "High quality" from the slides print setting

In the given example, the PUSH operation is performed by encapsulating a packet (IPv6, IPv4 or Layer 2 frame) into an outer IPv6 packet with a Segment Routing Header. Another possibility is to perform the \textit{insertion} of an SRH as a new header between the IPv6 header and the Next Header (e.g. the Trasport Layer Header, TCP or UDP), without encapsulating the packet in a new IPv6 packet. This option only applies to IPv6 packets and it is especially suited in case a host is acting as Source SR node (Headend node). 

In addition to the basic operations (PUSH/ NEXT/ CONTINUE), the \textit{SRv6 Network Programming} model \cite{id-srv6-network-prog} describes a set of functions that can be associated to segments and executed in a given SRv6 node. Examples of such functions are: different types of packet encapsulation (e.g. IPv6 in IPv6, IPv4 in IPv6, Ethernet in IPv6), corresponding decapsulation, lookup operation on a specific routing table (e.g. to support VPNs). A more complete list is provided in \cite{id-srv6-network-prog}, but the list is not meant to be exhaustive, as any function can be associated to a segment identifier in a node. Obviously, the definition of a standardized set of segment routing functions facilitates the deployment of SR domains with interoperable equipment from multiple vendors. 

According to \cite{id-srv6-network-prog}, we can revisit the notion of Segment IDentifier (SID) taking into account that IPv6 addresses are used as SIDs in SRv6. A 128 bit SID can be logically split in two fields and interpreted as LOCATOR:FUNCTION where LOCATOR includes the L most significant bits and FUNCTION the remaining 128-L least significant bits. The locator corresponds to an IPv6 prefix (e.g. with a length of 48 bits) that can be distributed by the routing protocols and provides the reachability of a node that hosts a number of functions. The length of the locator is not fixed and can be chosen by each operator for its own SR domain (also independently for different nodes). All the different functions residing in a node can share the same locator and have a different FUNCTION code, so that their SID will be different. From the routing point of view, the solution is very scalable as a single prefix is distributed, with limited impact on the routing tables of the nodes in the SR domain. The LOCATOR:FUNCTION representation of a SID can also be extended by splitting the FUNCTION field into FUNCTION:ARGUMENTS, so that the least significant bits can be used to provide information to a given function.

The concept of Network Programming consists in combining functions that can reside in different nodes to achieve \textit{a networking objective that goes beyond mere packet routing} \cite{id-srv6-network-prog}. The functions described in \cite{id-srv6-network-prog} can support valuable services and features like layer 3 and layer 2 VPNs, traffic engineering, fast rerouting. The Network Programming model offers the possibility to implement virtually any service by combining the basic functions in a \textit{network program} that can be embedded in the packet header. As shown in Fig.~\ref{fig:sr-header}, the SRH can include an optional section that carries Type Length Value (TLV) objects. These TLV objects can be defined to carry information that needs to be elaborated by one or more segments of an SR policy (Segment List). For example, the so-called HMAC TLV can be added and used to verify that an SRH header has been created by an authorized node and that the segment list is not modified in transit. Another potential use of TLV objects is for exchanging Operation and Maintenance (OAM) information among the nodes of the SR domain. 

% fast path vs slow path?

% My SID table ?

\subsection{Control plane for SR and relation with SDN}
\label{sec:sr_control_plane}

Control Plane operations are needed to complement the dataplane functionality and provide a complete solution for Segment Routing. The Control Plane can be based on a fully distributed approach, in which the routers are capable to take independent decisions to setup and enforce the SR Policies, it can rely on a centralized SR controller that takes decision and instructs the routers following the SDN principles, or on a combination of the two approaches (hybrid solution). 

For the SR-MPLS dataplane, the definition of a fully distributed approach has been worked out within the IETF, with the definition of extensions to the IGP routing protocols (OSPF, ISIS, see \cite{ietf-ospf-segment-routing-extensions} \cite{ietf-ospf-ospfv3-segment-routing-extensions} \cite{ietf-isis-segment-routing-extensions}). 
These extensions to the routing protocols are used by each routers to advertise the different types of IGP-segments (prefix, node, adjacency, anycast) and to distribute some SR configuration information. All other routers in the SR domain will receive this information by means of the IGP routing protocol. This information is needed to map the segments included in an SR policy into SIDs represented as MPLS labels in the SR-MPLS dataplane. As we have discussed in subsection \ref{sec:mpls-dataplane}, in the general case each router could allocate different ranges of labels to be used for Global Segments. The range of labels used for the global segments by a router, called \textit{SRGB - Segment Routing Global Block} is among the SR configuration information advertised using the routing protocol. We recall that it is strongly recommended to use an identical range of labels (SRGB) in all routers. 

For the IPv6 dataplane, the process of advertising the IGP-prefix, IGP-node and IGP-anycast segments is simplified thanks to the use of IPv6 addresses as SIDs. In particular, there is no need to extend the IGP routing protocols to distribute these segment types, represented as IPv6 prefixes that are natively distributed by the routing protocols. Also the definition of a Segment Routing Global Block as in the SR-MPLS is not needed and the operations related to Global Segments can rely on IPv6 addresses that are globally routable in the SR domain. This means that the Control Plane for SRv6 can use the regular IGP routing protocols (OSPF, ISIS) to support the basic operations, while extensions are still needed (\cite{id-isis-srv6-extensions} \cite{li-ospf-ospfv3-srv6-extensions}) to distribute IGP-Adjacency segments and other SR configuration information. 

The definition of the control plane for Segment Routing has started from the SR-MPLS dataplane and then the SRv6 dataplane has inherited most of the functionality, which has been adapted to the new dataplane. We observe that an original design goal of the control plane for Segment Routing has been to support the fully distributed approach, in which routers are capable to take autonomous decisions. This allows offering the the same functionality of a traditional MPLS network, which does not need a centralized SDN controller for its operations. On the other hand, we observe now a trend to focus on an hybrid approach, in which distributed routing protocols coexist with an SR controller. This hybrid approach is aligned with the vision of Software Defined Networking that aims at removing complexity from distributed nodes and to centralize control plane function in SDN controllers. In this light, the Segment Routing architecture can be deployed by seeking the right balance between distributed and centralized control. The distributed control is used by the routers to exchange reachability information and evaluate the shortest paths in a traditional way, with no need to interact with the centralized controller. We observe that this is the best approach to provide connectivity in Wide Area Networks in which the control connections between the nodes and the SDN controllers are affected by non-negligible latency and failure probability. Segment Routing can be used for Fast Reroute, by pre-configuring SR policies that provide alternate paths in case of link or node failures and are automatically activated by the node when the failure happens. The pre-calculation of such SR policies can be performed in a distributed mode or can be centralized in a controller. Basic topology information and additional information for Traffic Engineering need be conveyed to the controller, as well as service related information that is advertised by nodes using distributed routing protocols. The SDN controller can receive this information in different ways. For example, it can participate to the IGP routing protocol, it can interact with routers in a proprietary way to extract their IGP databases, it can receive information by routers using extensions to BGP-LS (BGP-Link State). Whatever mechanism has been used to retrieve the needed information from the nodes, the SDN controller is in charge to take decisions about the SR policies that implement advanced features or services like Traffic Engineering, VPNs or Service Function Chaining. This approach allows to clearly decouple the dataplane operations from the service logic operating in the control plane. The mechanisms and protocols for the SDN controller to enforce the SR policies by configuring the the nodes are left open in the SR architectural definition. As mentioned in \cite{rfc8402}, some options are Network Configuration Protocol (NETCONF), Path Computation Element Communication Protocol (PCEP) \cite{ietf-pce-segment-routing}, and BGP. The Openflow protocol can be used as a mechanism to configure the SR policies only for SR-MPLS, while SRv6 is not supported by the latest standard version of Openflow. An Open Source implementation of a SouthBound API for SRv6 based on gRPC is reported in \cite{ventre2018sdn}. The main characteristic of the Segment Routing solution compared to other SDN solutions is that only the edge nodes needs to be configured to enforce a given SR policy, while the internal nodes do not need to keep state per SR policy. This feature gives a substantial improvement in terms of scalability. 

\subsection{Segment Routing motivations and use cases}

As anticipated in the introduction section, the RFC 5439 \cite{rfc5439} has identified some scalability issues of traditional MPLS networks with Traffic Engineering support. These issues originated the interest in defining a more scalable solution like Segment Routing back in the late `00s. Several use cases and requirements for Segment Routing has been collected in a number of documents. In \cite{rfc7855}, the main use cases identified are: MPLS tunneling (i.e. to support VPN services), Fast ReRoute (FRR), Traffic Engineering (further classified in a number of more specific use cases). A set of Resiliency use cases is described in \cite{rfc8355}. In \cite{rfc8354}, the Segment Routing use cases for IPv6 networks are considered, with a set of exemplary deployment environments for SRv6: Small Office, Access Network, Data Center, Content Delivery Networks, Core Networks.

\begin{comment}
\begin{itemize}
      \item \cite{interconnecting} discusses the capability of SR-MPLS to interconnect a large number of end points with a limited SID space. 
\end{itemize}
\end{comment}




\input{sec/3-research}
\section{Standardization activities}
\label{sec:standard}
\begin{comment}

\begin{itemize}
    \item Describe all the standardization activities
    \item A first list can be found here http://www.segment-routing.net/ietf/ 
    \item Complete the section with the new drafts not listed above
\end{itemize}
\end{comment}
In this section we propose a classification and description of the standardization activity related to Segment Routing. We have classified  \numRFCStandardization Request For Comment (RFC) and \numDrafttandardization Internet Drafts. Our taxonomy is based on 7 categories and the result of the classification is shown in Table \ref{tab:standardization}. 

%{\begin{table}[] \begin{tabular}{l} \textbf{Protocol}  \\ \textbf{Extensions}  \end{tabular} \end{table}}
%{\parbox{1.5cm}{Long text to break}}

\begin{table}
\caption{\\Classification of the Standardization Activities documents}
\label{tab:standardization}
\begin{tabular}{|l|c|l|l|}
\hline
\multicolumn{3}{|l|}{\textbf{Architecture}}                                                                & \cite{rfc8402,id-segment-routing-policy,id-sr-policy-considerations,id-sr-policy-yang,id-segment-routing-mpls,id-srv6-network-prog,id-srv6-net-pgm-illustration,id-sr-service-programming} \\ \hline
\multicolumn{3}{|l|}{\textbf{Use-case and Requirements}}                                                   & \cite{rfc7855,rfc8355,rfc8354,id-segment-routing-msdc,id-segment-routing-central-epe,id-spring-large-scale-interconnect,id-sr-for-sdwan,id-srv6-mobile-uplane,id-srv6-mobile-pocs,id-network-slicing-building-blocks,id-sr-traffic-counters,id-spring-poi-sr} \\ \hline
\multicolumn{3}{|l|}{\textbf{Fast Reroute (FRR)}}                                                          & \cite{id-segment-routing-ti-lfa,id-segment-routing-uloop,id-microloop-avoidance} \\ \hline
\multicolumn{3}{|l|}{\textbf{OAM}}                                                                         & \cite{rfc8403,rfc8287,id-srv6-oam,id-sr-traffic-accounting,id-bfd-sr-policy} \\ \hline
\multicolumn{3}{|l|}{\textbf{Performance Measurement}}                                                     & \cite{id-sr-mpls-pm,id-udp-pm,rfc6374,rfc7876} \\ \hline
\multirow{8}{*}{\parbox{1.3cm}{\textbf{Protocol Extensions}}} & \multirow{2}{*}{\textbf{Data Plane}}    & \textit{SR-MPLS} & \cite{ietf-spring-segment-routing-ldp-interop,ietf-spring-mpls-anycast-segments,filsfils-spring-sr-recursing-info,rfc8426,ietf-mpls-sr-over-ip} \\ \cline{3-4} 
                                              &                                         & \textit{SRv6}    & \cite{ietf-6man-segment-routing-header,voyer-6man-extension-header-insertion,raza-spring-srv6-yang} \\ \cline{2-4} 
                                              & \multirow{6}{*}{\textbf{Control Plane}} & \textit{BGP}     & \cite{ietf-idr-bgp-prefix-sid,ietf-idr-segment-routing-te-policy,dawra-bess-srv6-services} \\ \cline{3-4} 
                                              &                                         & \textit{BGP-LS}  & \cite{ietf-idr-bgp-ls-segment-routing-ext,ietf-idr-te-lsp-distribution,ietf-idr-bgpls-segment-routing-epe,ietf-idr-bgp-ls-segment-routing-msd,rfc8571,ietf-idr-bgpls-srv6-ext,ketant-idr-bgp-ls-bgp-only-fabric,dawra-idr-bgp-ls-sr-service-segments} \\ \cline{3-4} 
                                              &                                         & \textit{IS-IS}   & \cite{ietf-isis-segment-routing-extensions,id-isis-srv6-extensions,ietf-lsr-flex-algo,id-lsr-flex-algo-yang,rfc8491,ietf-isis-l2bundles,rfc7810} \\ \cline{3-4} 
                                              &                                         & \textit{OSPF}    & 
                                              \begin{tabular}[c]{@{}l@{}}\cite{ietf-ospf-segment-routing-extensions,ietf-ospf-ospfv3-segment-routing-extensions,rfc8476,li-ospf-ospfv3-srv6-extensions}\\ \cite{ietf-lsr-flex-algo,id-lsr-flex-algo-yang,rfc7471}\end{tabular} \\ \cline{3-4} 
                                              &                                         & \textit{PCEP}    & \cite{ietf-pce-segment-routing,sivabalan-pce-binding-label-sid} \\ \cline{3-4} 
                                              &                                         & \textit{LISP}    & \cite{rodrigueznatal-lisp-srv6} \\ \hline
\end{tabular}
\end{table}

Hereafter we discuss the categories of the classification and then in the next subsection\extended{s} we report \shortver{an overview of the key standardization activities.}\extended{an overview of the key standardization activities and a detailed listing of all the documents.}
The first category is \textit{Architecture}, where all the documents regarding the description of the general architecture of a Segment Routing network are considered. The RFC 8402 \cite{rfc8402} falls into this category and describes the main features of SR, such as the source routing paradigm idea, the concept of SID and the definition of the two supported data planes.
In the category \textit{Use-case and Requirements} the documents describing use case scenarios for SR, e.g., use of SR in WANs, data center networks, mobility and network slicing, are inserted. Specifically, in this category there are 3 RFCs: i) RFC 7855 \cite{rfc7855} introducing the Source Packet Routing in Networking (SPRING), ii) RFC 8355 \cite{rfc8355} related to network resiliency using SR, and iii) RFC 8354 \cite{rfc8354} that describes how to steer IPv6 or MPLS packets over the SPRING architecture.
The third category is \textit{FRR} one, i.e, Fast Reroute realized through SR. The main standardization activity in this category is related to fast recovery after a link failure, and is referred to as Topology Independent Loop Free Alternate (TI-LFA), described in \cite{id-segment-routing-ti-lfa}. No RFC has been published in this category.
\textit{Operations, Administration, and Maintenance} (OAM) is the fourth defined category, where we include all the standardization activities related to tools used for maintenance of the network. As example, RFC 8287 \cite{rfc8287} focuses on the implementation of the ping and traceroute tools in SR.
In the \textit{Performance Measurement} category we consider all the documents describing measurement procedures related to performance parameters, such as delay and packet loss, in an SR network. We include in this category also the two RFCs 6374 \cite{rfc6374} and 7876 \cite{rfc7876} that explain how to measure delay and packet loss in MPLS. Despite these two documents have not been produced during the standardization activities of SR, we decided to include them in Table \ref{tab:standardization} since they are massively used in the drafts for performance monitoring regarding SR.
Finally, the \textit{Protocol Extensions} category covers two different set of documents related to extensions of legacy protocols: i) data plane protocols extensions, and ii) control plane protocols extensions.
As for the data plane, we include the two drafts describing SR-MPLS \cite{id-segment-routing-mpls} and SRv6 \cite{ietf-6man-segment-routing-header}.
As for the control plane, we the consider the documents on modifications to routing protocol (eg. BGP and OSPF) for the dynamic distribution of the SIDs in the SR network, or control protocol for the communication between a central controller (in case of centralized control plane) and the devices at the data plane (eg. PCEP).
%\shortver{In the next subsection we consider 9 key documents among the 70 included in the table and provide. have provided a selected select the provide an overview of the selected documents of the standardization activity.}
%\extended{In the next subsections we provide an overview of the selected documents of the standardization activity.}


\subsection{Key standardization efforts}
In this subsection, we provide an overview of the most important standardization efforts, by considering 9 documents among the almost 70 listed in Table \ref{tab:standardization}. \cite{rfc8402} and \cite{id-segment-routing-policy} define key tenets of the SR architecture and discuss the benefits brought by SR in terms of scalability, privacy and security. \cite{id-sr-service-programming}, \cite{id-sr-for-sdwan} and  \cite{id-srv6-mobile-uplane} elaborate more on the support of key use cases like NFV/SFC, SD-WAN and next generation of mobile networks. Instead, \cite{id-srv6-network-prog} extends basic SR concepts and \cite{id-segment-routing-ti-lfa} provides Fast Re-route (FRR) mechanisms against single failures. Finally,  
\cite{id-segment-routing-uloop} and \cite{ietf-lsr-flex-algo} analyzes the improvements of the routing stability and extensions to the routing protocols.

\cite{rfc8402} describes the Segment Routing architecture and its overall design. It defines the concept of a segment as a network instruction and presents the basic types of segments: prefix-SID, adjacency-SID, peering-SID and binding-SID. It also explains how such segments can be attached to data packets, leveraging the MPLS or IPv6 data planes, in order to steer traffic flows on any path in the network without requiring any per-flow state in the fabric.

\cite{id-segment-routing-policy} details the concept of an SR policy. It explains how Candidate Paths are defined as explicit SID-lists or as dynamically computed paths based on some optimization criteria, and how the active Candidate Path is selected. Moreover, it presents various ways of steering traffic into an SR Policy, automatically by coloring BGP service routes, remotely using a Binding-SID, or statically with route policies. The concepts described in this draft equally apply to the MPLS and SRv6 dataplanes.

The SR architecture is extended from the simple steering of packets across nodes to a general network programming approach in \cite{id-srv6-network-prog}. Using this framework, it is possible to encode arbitrary instructions and not only locations in a SID-list. Each SID is associated with a function to be executed at a specific location in the network. A set of basic functions are defined in~\cite{id-srv6-network-prog}, but other functions can be defined by network operators to fit their particular needs. Moreover, SID arguments allow functions to be provided additional context or their behavior to be tweaked on a per-flow basis.

\cite{id-sr-service-programming} defines the service SIDs and describes how to implement service programming in SR-MPLS and SRv6 enabled networks. Key tenet is to associate to a SID to each network function (either physical or virtual). These service SIDs may be combined together in a SID-list and finely programmed by leveraging the network programming concept. They can also be combined with other types of SIDs to provide traffic engineering or VPN services. Service segments can be associated to legacy appliances as well (with no SR capabilities), thanks to SR proxy mechanisms which perform the SR processing and hides the SR information from the network function.

In \cite{id-sr-for-sdwan} it is explained how SR technology enables underlay Service Level Agreements (SLA) to a VPN with scale and security while ensuring service opacity. SR based VPNs are analyzed in the case of a single provider and multiple providers. Moreover, the draft addresses also the control plane aspects of such solution which are managed by an over the top SD-WAN controller. Finally, the benefits brought by the SR technology to VPN services are analyzed in term of scalability, privacy, billing management and security.

\cite{id-srv6-mobile-uplane} describes the applicability of SRv6 to the user plane of mobile networks. Three modes are addressed: traditional mode, enhanced mode and enhanced mode with interworking. In the first, mobile user plane is unchanged except for the use of SRv6 as transport instead of GTP-U. Enhanced mode uses only SRv6 and its programming framework. Finally, the last mode uses SRv6 but provides also interworking functionality with legacy components still using GTP. Moreover, the document defines the SID functions for the SRv6 mobile user plane and describes a mechanism for end-to-end network slicing.

Topology Independent Loop-free Alternate (TI-LFA) \cite{id-segment-routing-ti-lfa} provides Fast Re-route (FRR) mechanisms protecting against link, node or local Shared Risk Link Groups (SRLGs)failures in SR enabled networks. For each destination in the network, a backup path is pre-computed and installed in the forwarding table, so that it is ready to be activated as soon as a failure is detected. For each destination, the backup path matches the post-convergence path, which is followed by the traffic after the network convergence. The draft analyzes also the benefits of using Segment Routing with the respect of traditional FRR solutions.

\cite{id-segment-routing-uloop} describes a mechanism leveraging SR to prevent transient routing inconsistencies during the convergence period that follows a network topology modification. Instead of directly converging to a new next-hop after a topology modification, a node can temporarily steer the impacted traffic through a set of loop-free SR Policies, thus preventing it from being affected by routing inconsistencies. After the network has fully converged, the temporary SR Policies are removed with no impact on the traffic.

\cite{ietf-lsr-flex-algo} defines a set of extensions to the IGP routing protocols that enable Prefix-SIDs to be associated with operator-defined shortest path algorithms, called SR Flexible Algorithms (Flex-Algo). These algorithms are defined as an optimization metric (IGP, TE or delay) and a set of constraints (e.g., resources to be excluded from the path). Each node participating in a Flex-Algo computes the shortest paths to the Prefix-SIDs of that Flex-Algo and installs them in it forwarding table. SR Flexible Algorithms allow traffic to be steered along traffic-engineered paths such as low-latency or dual-plane disjoint path with a single Prefix-SID.

\extended{\subsection{Architecture}
\begin{itemize}
    \item Segment Routing Architecture \cite{rfc8402} RFC 8402 
    \item SR Policy Architecture \cite{id-segment-routing-policy} WG DOCUMENT
    \item SR Policy Architecture - Companion document \cite{id-sr-policy-considerations} DRAFT
    \item YANG Data Model for Segment Routing Policy \cite{id-sr-policy-yang} DRAFT
    \item Segment Routing with MPLS data plane \cite{id-segment-routing-mpls} WG DOCUMENT
    \item SRv6 Network Programming \cite{id-srv6-network-prog} and \cite{id-srv6-net-pgm-illustration} DRAFT
    \item Segment Routing for Service Programming \cite{id-sr-service-programming} DRAFT
    \item IGP Flexible Algorithm \cite{ietf-lsr-flex-algo} and \cite{id-lsr-flex-algo-yang}
\end{itemize}}

\extended{\subsection{Use-Cases and Requirements}
\begin{itemize}
    \item Source Packet Routing in Networking (SPRING) Problem Statement and Requirements \cite{rfc7855} RFC 7855
    \item Resiliency Use Cases in Source Packet Routing in Networking (SPRING) Networks \cite{rfc8355} RFC 8355 \ste{it should be updated on sr.net!} 
    \item Use Cases for IPv6 Source Packet Routing in Networking (SPRING) \cite{rfc8354} RFC 8354
    \item BGP-Prefix Segment in large-scale data centers \cite{id-segment-routing-msdc} WG DOCUMENT
    \item Segment Routing Centralized BGP Peer Engineering \cite{id-segment-routing-central-epe} WG DOCUMENT (Submitted to IESG for Publication)
    \item Interconnecting Millions Of Endpoints With Segment Routing \cite{id-spring-large-scale-interconnect} DRAFT
    \item SR for SDWAN - VPN with Underlay SLA \cite{id-sr-for-sdwan} DRAFT
    \item Segment Routing IPv6 for Mobile User Plane \cite{id-srv6-mobile-uplane} DRAFT \ste{the name could be updated on sr.net}
    \item SRv6 for Mobile User-Plane 3GPP STUDY ITEM \ste{There is not a 3GPPP document with this name, there is this document ``Study on User-plane Protocol in 5GC'' which is still empty (Specification \#: 29.892)}
    \item Segment Routing IPv6 for mobile user-plane PoCs \cite{id-srv6-mobile-pocs} DRAFT \ste{the name could be updated on sr.net}
    \item Building blocks for Slicing in Segment Routing Network \cite{id-network-slicing-building-blocks} DRAFT
    \item Segment Routing Traffic Accounting Counters \cite{id-sr-traffic-counters} DRAFT
    \item Packet-Optical Integration in Segment Routing \cite{id-spring-poi-sr} DRAFT
\end{itemize}}

\extended{\subsection{Fast Reroute (FRR)}
\begin{itemize}
    \item Topology Independent Fast Reroute using Segment Routing \cite{id-segment-routing-ti-lfa} DRAFT
    \item Loop avoidance using Segment Routing \cite{id-segment-routing-uloop} DRAFT
    \item Micro-loop avoidance using SPRING \cite{id-microloop-avoidance} DRAFT EXPIRED \ste{do you want to mark expired drafts on sr.net?}
\end{itemize}}

\extended{\subsection{OAM}
\begin{itemize}
    \item A Scalable and Topology-Aware MPLS Dataplane Monitoring System \cite{rfc8403} RFC 8403
    \item Label Switched Path (LSP) Ping/Traceroute for Segment Routing (SR) IGP-Prefix and IGP-Adjacency Segment Identifiers (SIDs) with MPLS Data Planes \cite{rfc8287} \ste{update the name on sr.net?} RFC 8287
    \item Operations, Administration, and Maintenance (OAM) in Segment Routing Networks with IPv6 Data plane (SRv6) \cite{id-srv6-oam} \ste{update the name on sr.net?} DRAFT
    \item Traffic Accounting in Segment Routing Networks \cite{id-sr-traffic-accounting} DRAFT \ste{update the name on sr.net?}
    \item BFD for SR Policies \cite{id-bfd-sr-policy} DRAFT
\end{itemize}}

\extended{\subsection{Performance Measurement}
\begin{itemize}
    \item Packet Loss and Delay Measurement for MPLS Networks \cite{rfc6374} RFC 6374 \ste{is it a Segment Routing specification?} \plv{no}
    \item UDP Return Path for Packet Loss and Delay Measurement for MPLS Networks \cite{rfc7876} RFC 7876 \pier{no}
    \item Performance Measurement in Segment Routing Networks with MPLS Data Plane \cite{id-sr-mpls-pm} DRAFT
    \item UDP Path for In-band Performance Measurement for Segment Routing Networks \cite{id-udp-pm} DRAFT
\end{itemize}}

\extended{\subsection{Protocol Extensions}
\begin{itemize}
    \item IGP Flexible Algorithm WG DOCUMENT \cite{ietf-lsr-flex-algo} and \cite{id-lsr-flex-algo-yang}
\end{itemize}
\subsubsection{SR-MPLS}
\begin{itemize}
    \item Segment Routing interworking with LDP WG DOCUMENT \cite{ietf-spring-segment-routing-ldp-interop}
    \item Anycast Segments in MPLS based Segment Routing DRAFT \cite{ietf-spring-mpls-anycast-segments} 
    \item Segment Routing Recursive Information DRAFT \cite{filsfils-spring-sr-recursing-info}
    \item Recommendations for RSVP-TE and Segment Routing LSP co-existance \cite{rfc8426} RFC 8426
    \item SR-MPLS over IP DRAFT \cite{ietf-mpls-sr-over-ip}
\end{itemize}
\subsubsection{SRv6}
\begin{itemize}
    \item IPv6 Segment Routing Header (SRH) WG DOCUMENT \cite{ietf-6man-segment-routing-header}
%    replaces \cite{previdi-6man-segment-routing-header-08}
    \item Insertion of IPv6 Segment Routing Headers in a Controlled Domain DRAFT \cite{voyer-6man-extension-header-insertion}
    \item YANG Data Model for SRv6 DRAFT \cite{raza-spring-srv6-yang}
\end{itemize}    
\subsubsection{BGP}
\begin{itemize}
    \item Segment Routing Prefix SID extensions for BGP WG DOCUMENT \cite{ietf-idr-bgp-prefix-sid}
    \item Advertising Segment Routing Traffic Engineering Policies in BGP WG DOCUMENT \cite{ietf-idr-segment-routing-te-policy}
    \item BGP Signaling of IPv6-Segment-Routing-based VPN Networks DRAFT \cite{dawra-bess-srv6-services}
\end{itemize}    
\subsubsection{BGP-LS}
\begin{itemize}
    \item BGP Link-State extensions for Segment Routing WG DOCUMENT \cite{ietf-idr-bgp-ls-segment-routing-ext}
    \item SR Policy Distribution via BGP-LS WG DOCUMENT \cite{ietf-idr-te-lsp-distribution}
    \item Segment Routing BGP Egress Peer Engineering BGP-LS Extensions WG DOCUMENT \cite{ietf-idr-bgpls-segment-routing-epe}
    \item Signaling Maximum SID Depth using Border Gateway Protocol Link-State WG DOCUMENT \cite{ietf-idr-bgp-ls-segment-routing-msd}
    \item BGP - Link State (BGP-LS) Advertisement of IGP Traffic Engineering Performance Metric Extensions RFC 8571 \cite{rfc8571}
    \item BGP Link State extensions for IPv6 Segment Routing (SRv6) DRAFT \cite{ietf-idr-bgpls-srv6-ext}
    \item BGP Link-State Extensions for BGP-only Fabric DRAFT \cite{ketant-idr-bgp-ls-bgp-only-fabric}
    \item BGP-LS Advertisement of Segment Routing Service Segments \cite{dawra-idr-bgp-ls-sr-service-segments}
\end{itemize}    
\subsubsection{IS-IS}
\begin{itemize}
    \item IS-IS Extensions for Segment Routing WG DOCUMENT \cite{ietf-isis-segment-routing-extensions}
    \item IS-IS Extensions to Support Segment Routing over IPv6 Dataplane DRAFT \cite{id-isis-srv6-extensions}
    \item Signaling MSD (Maximum SID Depth) using IS-IS RFC 8491 \cite{rfc8491}
    \item Advertising L2 Bundle Member Link Attributes in IS-IS WG DOCUMENT \cite{ietf-isis-l2bundles}
    \item IS-IS Traffic Engineering (TE) Metric Extensions RFC 7810 \cite{rfc7810}
\end{itemize}    
\subsubsection{OSPF}
\begin{itemize}
    \item OSPF Extensions for Segment Routing WG DOCUMENT \cite{ietf-ospf-segment-routing-extensions}
    \item OSPFv3 Extensions for Segment Routing WG DOCUMENT \cite{ietf-ospf-ospfv3-segment-routing-extensions}
    \item Signaling MSD (Maximum SID Depth) using OSPF RFC 8476 \cite{rfc8476}
    \item OSPFv3 Extensions for SRv6 DRAFT \cite{li-ospf-ospfv3-srv6-extensions}
    \item OSPF Traffic Engineering (TE) Metric Extensions RFC 7471 \cite{rfc7471}
\end{itemize}    
\subsubsection{PCEP}
\begin{itemize}
    \item PCEP Extensions for Segment Routing WG DOCUMENT \cite{ietf-pce-segment-routing}
    \item Carrying Binding Label/Segment-ID in PCE-based Networks DRAFT \cite{sivabalan-pce-binding-label-sid}
\end{itemize}    
\subsubsection{LISP}
\begin{itemize}
    \item LISP Control Plane for SRv6 Endpoint Mobility DRAFT \cite{rodrigueznatal-lisp-srv6}
\end{itemize}} 
\section{SR implementations and deployments}
\label{sec:tools}

\begin{comment}

\begin{itemize}
    \item VPP implementation of SR
    \item Other open source implementations from the works listed in Research Directions (if the tools are open source and are valuable)
    \item Other implementations from vendors
\end{itemize}

\end{comment}

In this section, we describe the implementation results related to SR. We will mostly focus on the SRv6 version which is attracting a lot of interest and development efforts. The SR-MPLS version is already in a mature development status, well supported by the main core router vendors (e.g. Cisco, Huawei, Juniper). SR-MPLS can be incrementally deployed in current IP-MPLS backbones, as it only requires software updates to networking devices. Operators can migrate to SR-MPLS to simplify the control plane operations and improve the scalability. As for the SRv6 dataplane, there are two main Open Source dataplane implementations for software routers: the Linux kernel implementation (described in Subsection \ref{sec:linux}) and the realization done inside the FD.io VPP project (described in Subsection \ref{sec:vpp}).  Section \ref{sec:rest} elaborates more on the implementation results related to the research activities. Finally, in Subsection \ref{sec:hw_interop} we briefly analyze the hardware implementations of SR, the inter-operability efforts done by several vendors and the currently planned deployments of SRv6 in large production networks.

\subsection{Linux kernel}
\label{sec:linux}

The SRv6 capabilities were first added in Linux kernel 4.10 \cite{lebrun2017implementing}. Kernel 4.10 includes the support for some SRv6 \textit{transit} behaviors (e.g., \textit{T.Insert} and \textit{T.Encaps}). The \textit{transit} behaviors are implemented as Linux Lightweight Tunnel (LWT). The implementation of the iproute2~\cite{iproute2} user space utility is extended to support adding a \textit{localsid} associated with an SRv6 \textit{transit} behavior~\cite{srv6-impl-basic}. SRv6 \textit{localsids} with \textit{transit} behavior are added as IPv6 FIB entries into the kernel main routing table. Kernel 4.14 is another important milestone for the SRv6 support in Linux: a set of SRv6 \textit{endpoint} behaviors have been implemented by adding a new type of LWT~\cite{lebrun2017reaping}. The supported SRv6 \textit{endpoint} behaviors are \textit{End.X}, \textit{End.T}, \textit{End.DX2}, \textit{End.DX4}. \textit{End.DX6}, \textit{End.DT6}, \textit{End.B6}, and \textit{End.B6.Encaps}. Some new \textit{transit} behaviors have been added (e.g., \textit{T.Encaps.L2}). The iproute2 implementation was extended as well \cite{iproute2} \cite{srv6-impl-adv}. The SRv6 capabilities in Linux kernel were extended in kernel 4.16 \cite{kernel4-16} to include the netfilter framework \cite{netfilter}. A new iptables match extension, named \texttt{srh}, was added to the kernel to support matching of SRH fields. The \texttt{srh} match extension is a part of the SERA firewall~\cite{paper-sera} and supports matching all the fields of the SRH. The implementation of iptables user space utility \cite{wiki-iptables} is extended with a new shared library (\texttt{libip6t\_srh}) that allows to define iptables rules with \texttt{srh} options. Kernel 4.18 \cite{kernel4-18} has seen some more features both in the core SRv6 stack and the netfilter framework. 

In the netfilter framework, the \texttt{srh} match is extended to provide the matching of SRH's \textit{Previous SID}, \textit{Next SID}, and \textit{Last SID}. The iptables user space utility is updated as well to support the new matching options. Instead, a new feature is added in the Linux SRv6 stack to support custom SRv6 network functions implemented as small eBPF \cite{lwn-ebpf} programs. \cite{xhonneux2018leveraging} extends \cite{id-srv6-network-prog} introducing a new \textit{End} behavior the so called \textit{End.BPF}. From an implementation point of view a new hook for BPF is added to the SRv6 infrastructure that can be used by network operators to attach small programs written in \textit{C} to SRv6 SIDs which have direct access to the Ethernet frames. Moreover, specific SRv6-BPF helpers have been provided in order to allow \textit{End.BPF} functions to execute basic SRv6 actions (\textit{End.X}, \textit{End.T} and many others) or adding TLVs. This allows also to implement custom SRv6 \textit{transit} behaviors (mainly to extend SRv6 encapsulation policies implemented by the kernel). The tutorial about eBPF extensions to SRv6 is available at \cite{srv6-ebpf-tutorial}. The source code of the sample applications described in \cite{xhonneux2018leveraging} is freely available at \cite{srv6-ebpf-code1}. Instead, the eBPF-based fast-reroute and failure detection schemes described in \cite{xhonneux2018flexible} is available at \cite{srv6-ebpf-code2}.

SR-MPLS has not received the same attention of the SRv6 implementation in the Linux kernel. All the features which are available are mostly related to the well-established MPLS forwarding. They have been made available from the version 4.1 of the kernel. In particular, kernel v4.1 has seen the introduction of the MPLS Label Switching Router (LSR) behavior. MPLS capabilities have been extended later in the kernel v4.3. LWT framework and MPLS tunnel were added allowing the implementation of the MPLS Label Edge Router (LER) behavior. Finally, MPLS multipath functionality has been added only in the version 4.5 of the kernel.

In general, the Linux kernel lacks of the support of the SR policy framework which is instead available for FD.io VPP implementation (Subsection \ref{sec:vpp}). This means that at the time of writing is not possible to create a SR policy (both MPLS and IPv6) and associate a \textit{BindingSID} to it nor instantiate SR-MPLS/SRv6 steering rules pointing to SR-MPLS/SRv6 policies.

\subsection{FD.io VPP}
\label{sec:vpp}

FD.io VPP \cite{fd-io-vpp} 17.04 included the support for the \textit{transit} behaviors and most of the \textit{endpoint} behaviors defined in \cite{id-srv6-network-prog}. These behaviors are implemented in dedicated VPP graph nodes. The SRv6 graph nodes perform the required SRv6 behaviors as well the IPv6 processing (e.g. decrements Hop Limit). Whenever an SRv6 segment is instantiated, a new IPv6 FIB entry is created for the segment address pointing to the corresponding VPP graph node. Release 17.04 also brought SR headend capabilities to VPP by introducing the concept of SR policy in the SRv6 implementation. In VPP, an SR policy is uniquely identified by its \textit{BindingSID} address, which serves as a key to a particular SR policy. This is not compliant with the SR policy definition \cite{id-segment-routing-policy}, but a reasonable shortcut considering the absence of control-plane capabilities in VPP. 

The SR policies in VPP support several SID lists with weighted load-balancing of the traffic among them. When a new segment list is specified for an SR policy, VPP pre-computes the rewrite string that will be used upon steering traffic into that SID list, either via a \textit{transit} behavior or a \textit{BindingSID}. VPP then initializes one FIB entry for the SR policy \textit{BindingSID} in the FIB and an entry in a hidden FIB table for the IPv6 traffic steered into the SR policy via a \textit{transit} behavior. Each one of these FIB entries points to the SR policy object, which in turn recurses on the weighted segment lists.

Traffic can be steered into an SR policy either by sending it to the corresponding \textit{BindingSID} or by configuring a rule, called steering policy, that directs all traffic transiting towards a particular IP prefix or L2 interface into a SRv6 policy. The latter mechanism is implemented as FIB entry for the steered traffic in the main FIB to be resolved via the FIB entry of the SR policy in the hidden FIB table. In this way, a hierarchical FIB structure is realized: the traffic is not directly steered over an SR policy, but instead directed to a hidden FIB entry associated with the policy. This allows the SR policy to be modified without requiring any change to the steering rules that point towards it.

Release 17.04 has also seen the introduction of the SRv6 LocalSID development framework and the SR-MPLS implementation. The former is an API which allows developers to create new SRv6 \textit{endpoint} behaviors using the VPP plugin framework. The principle is that the developer only codes the actual behavior, i.e. the VPP graph node. Instead, the segment instantiation, listing and removal are performed by the existing SRv6 code. The SR-MPLS framework is focused on the SR policies, as well on its steering. Likewise in SRv6, an SR policy is defined by a MPLS label representing the \textit{BindingSID} and a weighted set of MPLS stacks of labels. Spray policies are a specific type of SR-MPLS policies where the packet is replicated on all the SID lists, rather than load-balanced among them. Tto steer packets in transit into an SR-MPLS policy, the user has to to create an SR-MPLS steering policy. Instead, others SR-MPLS features, such as for example adjacency SIDs, can be achieved using the regular VPP MPLS implementation. In release 18.04, service programming proxy behaviors \textit{End.AS}, \textit{End.AD} and \textit{End.AM} were introduced as VPP plugins leveraging the framework described before.

\subsection{Rest of us}
\label{sec:rest}

Most of the research efforts analyzed in this work have released as open source the components and the extensions realized for SR. Some of the them build upon the implementation described in the previous Subsections, while other propose alternative solutions. The SREXT module (\cite{implementationof}) provides a complementary implementation of SRv6 in Linux based nodes. When it was designed, the Linux kernel only offered the basic SRv6 processing (\textit{End} behavior). SREXT complemented the SRv6 Linux kernel implementation providing a set of behaviors that were not supported yet. Currently most of the behaviors implemented in SREXT are supported by the mainline of Linux kernel (with the exception of the SR proxy behaviors). SREXT provides an additional local SID table which coexists with the one maintained by the Linux kernel. The SREXT module registers itself as a callback function in the \textit{pre-routing} hook of the netfilter \cite{netfilter} framework. Since its position is at beginning of the netfilter processing, it is invoked for each received IPv6 packet. If the destination IPv6 address matches an entry in the local SID table, the associated behavior is applied otherwise the packet will follow the normal processing of the routing subsystem. The source code of SREXT together with the Vagrant box are available at \cite{srext-home} which allow to bootstrap a small testbed in few minutes and experiment with SREXT features.

FRRouting (FRR) \cite{frr} is an open source routing protocol stack for Linux forked from Quagga \cite{quagga}. In FRR, there is an experimental support \cite{frr-sr} of the draft \cite{ietf-ospf-segment-routing-extensions} which defines the OSPFv2 extensions for Segment Routing (SR-MPLS). At the time of writing, there is no support for SRv6.

The SPRING-OPEN project \cite{springopen} provides an SDN-based implementation of SR-MPLS. The architecture is based on a logically centralized control plane, built on top of ONOS. Part of this work converged later in the Trellis project \cite{trellis}, an SDN based leaf-spine fabric which has been designed to be multi-purpose and multi-vendor. Trellis supports also P4/P4Runtime \cite{p4} \cite{p4runtime} devices as well as Stratum \cite{stratum} enabled devices. Trellis has been used as underlay/overlay fabric in the CORD project \cite{cord} which aims at redesigning central-office architectures. Recently, it has been integrated in the SEBA project \cite{SEBA} which targets residential-access networks. All the software stack and the documentation is freely available on \cite{trellis}. Moreover, a tutorial together with a ready-to-go VM can be downloaded from \cite{trellis-tutorial}.

PMSR (\cite{pmsr} and \cite{trafficpmsr}) provides an open source implementation of SR-MPLS together with the realization of a SDN control plane. The data plane leverages the OSHI architecture (\cite{oshi1}, \cite{oshi2}) which combines a SDN data plane, implemented with Open vSwitch \cite{ovs}, and OSPFv2 control logic, realized with Quagga. This architecture is extended in PMSR with the introduction of a Routes Extraction entity which connects to Quagga and receives routes update using the FPM interface provided by Quagga \cite{quagga}. These routes are then translated in SIDs and installed in the SDN data plane as OpenFlow MPLS forwarding rules. Authors provide a set of management tools \cite{mantoo} which assist experimenters and relieve them from a huge configuration effort. A tutorial to start working with PMSR is available on \cite{pmsr-tutorial}; instead a ready-to-go VM with all the dependencies installed can be downloaded from \cite{oshi-home}.

Software Resolved Network (SRN) (described in \cite{lebrun2018software} and \cite{duchene2018exploring}) is a variant of the SDN architecture. The network controller is logically centralized and co-located with a DNS resolver and uses extensions of the DNS protocol to interact with end-hosts. OVSDB \cite{rfc7047} is used to enable the communication between SDN controller and the network nodes: i) the latter populates the distributed database with the topology information and TE metadata; ii) the former once computed the path, upon a request, populates the OVSDB instance with the SRv6 Segment list matching the desired requirements. Finally, this is pulled by the access device which enables the communication of the end-hosts. The source code is freely available at \cite{srn-code}. An overview of the architecture can be found in \cite{srn-overview}. A ready-to-go VM with packaged experiments can be created using the instructions in \cite{srn-vm}.

\cite{ventre2018sdn} proposes a classical SDN architecture for SRv6 technology: a centralized logic takes decisions on the Segment Lists that need to be applied to implement the services, then the SDN controller, using a southbound API, interacts with the SR enabled devices to enforce the application of such Segment Lists. The code related to the SDN architecture, i.e. the four different implementations of the Southbound API and the topology discovery, can be downloaded from the page of the project \cite{srv6-sdn}. In addition the authors, to support both the development and testing aspects, have realized an Intent based emulation system to build realistic and reproducible experiments relieving the experimenters from a huge configuration effort. The emulation tools are available at \cite{rose}.

SRV6Pipes \cite{duchene2018srv6pipes} is an extension of the SRv6 implementation in the Linux kernel \cite{lebrun2017implementing} which enables chaining and operation of in-network functions operating on streams. SRv6 policies are installed using the SRN architecture \cite{lebrun2018software} described earlier in this section. However, SRN components are not mandatory since SRv6 policies can be installed in the edge nodes using the \textit{iproute} utility. SRv6Pipes is composed by multiple components that necessarily need to be installed in the target machines: TCP proxy, patched Kernel and user space utilities. The minimum components can be downloaded from the repository of the project \cite{srv6pipes-code}. The complete code of the experiments together with a walkthrough can be found in \cite{srv6pipes}.

SRNK \cite{mayer2019efficient} extends the implementation of SRv6 in the Linux kernel \cite{lebrun2017implementing} adding the support for the \textit{End.AS} and \textit{End.AD} behaviors. The source code is freely available at \cite{srnk-home}, where it is possible to download the patched Linux kernel (starting from 4.14.0 branch) and the patched iproute2 (starting from iproute2-ss171112 tag). Instead, the detailed configurations steps of the SR-proxy are reported in the Appendices A and B of \cite{mayer2019efficient}.

\cite{aubry2018robustly} describes the implementation of a path computation element able to compute robust disjoints SR paths which remains disjoint even after an input set of failures without the need of configuration changes. The java implementation of the algorithms, the public topologies used for the experiments, the experimental results and a detailed walkthrough to replicate the experiments of the paper are available at \cite{robustly-home}.

\subsection{Hardware implementations, inter-operability efforts and planned deployments}
\label{sec:hw_interop}

\cite{srv6-inter-op} provides an overview of IPv6 Segment Routing implementations and details some interoperability scenarios that have been demonstrated at public events. With regards to the hardware implementations, the platforms ASR 1000, ASR 9000, NCS 5500, NCS 540 and NCS560 are reported as the Cisco Routing platforms supporting SRH processing \cite{cisco}; ASR 9000 and NCS 5500 being deployed in production networks. The programmable devices based on Tofino chipset \cite{barefoot} can be programmed to support SRH processing. This is also true for the reference software implementation of the P4 devices \cite{bmv2}, Stratum based devices \cite{stratum} and all the programmable chipset (Cavium Xpliant \cite{cavium} to give an example). The draft elaborates also on the open source applications supporting the processing of the IPv6 Segment Routing header, among which we mention the well known Wireshark \cite{wireshark}, tcpdump \cite{tcpdump}, iptables \cite{iptables}, nftables \cite{nftables} and snort \cite{snort}.

The implementations, described in the previous paragraph, have been used in interoperability testing scenarios showcased at the 2017 SIGCOMM conference \cite{interop-demo}. The set of experiments included a L3 VPN scenario augmented with TE functionality and services function chaining processing. SREXT, VPP, Linux kernel, Barefoot Tofino, Cisco NCS5500 and Cisco ASR1000 routers were the network devices implementing SRv6 behaviors. While iptables and snort have been used as service functions. Finally, Wireshark and tcpdump have been leveraged to verify the proper operations of the network. 

\cite{ietf-6man-segment-routing-header} reports the implementation status of other vendors. Juniper's Trio and vTrio NPUs has an experimental support of SRH (SRH insertion mode and \textit{End} processing of interfaces addresses). Instead, Huawei's VRP platform is in production stage and has the capability of adding SRH header and performing \textit{End} processing.

To conclude the section, we mention that large scale deployments of SRv6 in production networks have been recently announced. In particular two mobile operators are planning SRv6 deployments in their ``pre-5G'' or ``5G-ready'' networks (\cite{blog-cisco-softbank-srv6,blog-cisco-iliad-srv6}).

%Stefano: in the submitted version I have removed the references to the blog posts from CISCO:
%To conclude the section, we mention that large scale deployments of SRv6 in production networks have been recently announced. In particular two mobile operators are planning SRv6 deployments in their ``pre-5G'' or ``5G-ready'' networks\shortver{.} \extended{(\cite{blog-cisco-softbank-srv6,blog-cisco-iliad-srv6}).}



\begin{comment}
\section{On-going deployments}
\label{sec:deployments}

\begin{itemize}
    \item Briefly describe the on-going deployments
\end{itemize}
\section{Future research directions}
\label{sec:future}

\begin{itemize}
    \item Describe future research directions
\end{itemize}
\end{comment}

\extended{\section{Conclusion}
\label{sec:conclusion}

\begin{itemize}
    \item Describe future research directions
    \item Wrap-up
\end{itemize}}

\section*{Acknowledgment}
    This work has received funding from the Cisco University Research Program Fund.

\bibliographystyle{IEEEtran}
\bibliography{references}

\extended{
\vspace{-4em}
\begin{IEEEbiography}[{\includegraphics[width=1in,height=1.25in,clip,keepaspectratio]{fig/bio/ventre.jpg}}]{Pier Luigi Ventre}
received his Master's degree in Computer Engineering from University of Rome Tor Vergata in 2014, with a thesis on Information-Centric Networking and Software Defined Networking. From 2013 to 2015, he was one of the beneficiary of the scholarship ``Orio Carlini'' granted by the Italian NREN GARR. His main research interests focus on Computer Networks, Software Defined Networking, Virtualization and Information-Centric Networking. He worked in several EU research projects. Currently, He is a PhD student in Electronic Engineering at University of Rome Tor Vergata.
\end{IEEEbiography}
\vspace{-4em}
\begin{IEEEbiography}[{\includegraphics[width=0.9in,height=1.10in,clip,keepaspectratio]{fig/bio/salsano.png}}]{Stefano Salsano}
(M'98-SM'13) received his PhD from University of Rome ``La Sapienza'' in 1998. He is Associate Professor at the University of Rome Tor Vergata. He participated in 15 research projects founded by the EU, being project coordinator in one of them and technical coordinator in two. He has been PI in several research and technology transfer contracts funded by industries. 
His current research interests include SDN, Network Virtualization, Cybersecurity, Information Centric Networking. He is co-author of an IETF RFC and of more than 140 peer-reviewed papers and book chapters.
\end{IEEEbiography}
\begin{IEEEbiography}[{\includegraphics[width=1in,height=1.25in,clip,keepaspectratio]{fig/bio/filsfils.jpg}}]{Clarence Filsfils}
is a Cisco Systems Fellow, has a 20-year expertise leading innovation, productization, marketing and deployment for Cisco Systems. He invented the Segment Routing Technology and is leading its productization, marketing and deployment. Previously, he invented and led the Fast Routing Convergence Technology and was the lead designer for Cisco System's QoS deployments. Clarence is a regular speaker at leading industry conferences. He holds over 130 patents and is a prolific writer, either in academic circle, or standardization or books.
\end{IEEEbiography}}
\end{document}

