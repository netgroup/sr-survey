\section{Conclusion}
\label{sec:conclusion}

The Segment Routing technology is based on the source routing and tunneling paradigms. Segment Routing supports services like Traffic Engineering, Virtual Private Networks, Fast Restoration in IP backbones and datacenters and proved to be flexible in supporting new use cases. Moreover, the SR architecture reduces the amount of state information that needs to be configured in the core nodes. 

SR-MPLS and SRv6 are the two data plane instantiations of the SR architecture. This is the first tutorial and survey work covering in detail the novel SRv6 solution (i.e. SR over IPv6 data plane), which represents the most promising implementation for future research activity. SRv6 provides a consistent solution for solving long-lived problems in the IP networks, simplifying protocol stacks and improving scalability with respect to current solutions.

In the survey we have covered the standardization work and the research activities related to Segment Routing. We also considered the recent deployments of SR in real networks and the existing SR implementations, with focus on the open source tools that can support SR research and development activities.

\begin{comment}
To the best of our knowledge, there is only another survey about Segment Routing, namely \cite{abdullah2018segment}, which focuses on SR-MPLS and only mentions SRv6 architecture and lacks of an analysis of the standardization efforts and implementation results.
\end{comment}

As for the research activities, we covered 90 scientific papers related to SR and proposed a taxonomy for their classification. One of the main outcome of the classification was to identify relationships between the SR features and the research topics. For instance, the source routing paradigm has turned to be the key enabling feature for the implementation of Traffic Engineering solutions in an SR network, while the routing flexibility feature is mainly used to realize network monitoring tools. We also identified the most interesting SR standardization documents and provided a taxonomy for their classification.

The review of the SR implementations highlighted the maturity of the open source solutions based on Linux kernel and VPP. Linux and VPP, widely used by the research and developer community, allow to easily deploy a virtual SR playground. 

We hope that this tutorial and survey work will further increase the attention of the research community on the Segment Routing technology and motivate new researchers to join the development of new use cases and the standardization efforts. We have anticipated future research directions which can be taken as starting points.

New versions of this survey will be available at \cite{ventre2019survey}. Moreover, we strongly encourage the community to provide feedback and updates as new research works and SR deployments come out and technology evolves. For this reason we have created a public repository\footnote{https://github.com/netgroup/sr-survey}, where interested researchers can contribute and update this documentation.