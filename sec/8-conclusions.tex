\section{Conclusion}
\label{sec:conclusion}

The Segment Routing technology is based on the source routing and tunneling paradigms. Segment Routing supports services like Traffic Engineering, Virtual Private Networks, Fast Restoration in IP backbones and datacenters and proved to be flexible in supporting new use cases. Moreover, the SR architecture reduces the amount of state information that needs to be configured in the core nodes. 

In the survey we investigated research works and standardization activity related to the Segment Routing paradigm. We also provided a detailed evaluation of real implementations, focusing on available tools to realize experimental testbed for SR research and development activities.

SR-MPLS and SRv6 are the two dataplane instantiations of the SR architecture. This is the first tutorial and survey work covering in detail the novel implementation of SR over IPv6, i.e. SRv6, that represents the most promising implementation for future research activity. SRv6 provides a consistent solution for solving long-lived problems in the IP networks, simplifying protocol stacks and improving scalability with respect to current solutions.

\begin{comment}
To the best of our knowledge, there is only another survey about Segment Routing, namely \cite{abdullah2018segment}, which focuses on SR-MPLS and only mentions SRv6 architecture and lacks of an analysis of the standardization efforts and implementation results.
\end{comment}

We covered more than 75 scientific papers related to SR and we proposed a taxonomy for their classification. One of the main outcome of the classification was to identify relationships between the SR features and the research topics. For instance, the source routing paradigm has turned to be the key enabling feature for the implementation of Traffic Engineering solutions in an SR network, while the routing flexibility feature is mainly used to realize network monitoring tools. We also identified the most interesting SR standardization documents, providing a taxonomy for their classification. The study on the SR implementations highlighted the maturity of the solutions based on Linux kernel and VPP. Linux and VPP, widely used by the research and developer community, allow to easily deploy a virtual SR playground. 

To conclude, we try to provide guidelines for future directions of the SR research activities. To begin with, we believe that more emphasis should be given to the Programmability features of SR, that can be fruitfully exploited in the contexts of NFV and SFC. In this regard, new abstraction models for the control of Segment Routing network could be studied. Another interesting the topic is related to moving SR functions from the end-host stacks closer to the hardware. SmartNICs allow to implement network traffic processing on the NIC instead of using the CPU of the end-nodes. There is also a growing interest in the integration of the SRv6 technology into Cloud orchestrators like OpenStack \cite{openstack} and Kubernetes \cite{kubernetes}. The goal is to replace the current dataplane mechanism based on legacy tunneling mechanisms like VXLAN with a drastic simplification of the network stack. Moreover, the use of SR in the 5G networks and IoT is a promising research topic, above all in conjunction with SRv6. Finally, there are a lot of efforts in supporting field deployments as witnessed in several recent announcements \cite{srnews}. 

We hope that this tutorial and survey work will raise the attention of the research community on the Segment Routing technology and motivate new researchers to join the development of new use cases and the standardization efforts. We have anticipated new challenges and future topics earlier in this section which can be picked as starting points.

New versions of this survey will be available at \cite{ventre2019survey}. Moreover, we strongly encourage the community to provide feedback and updates as new research works and SR deployments come out and technology evolves. For this reason we have created a public repository\footnote{https://github.com/netgroup/sr-survey}, where interested people can contribute and update this documentation.