\section{Introduction}
\label{sec:intro}

\IEEEPARstart{S}egment Routing (SR) is based on the loose Source Routing concept. A node can include an ordered list of instructions in the packet headers. These instructions steer the forwarding and the processing of the packet along its path in the network. The single instructions are called \textit{segments}, a sequence of instructions can be referred to as a \textit{segment list} or as an \textit{SR Policy} . Each segment can enforce a topological requirement (e.g. pass through a node or an interface) or a service requirement (e.g. execute an operation on the packet). The term \textit{segment} refers to the fact that a network path towards a destination can be split in segments by adding intermediate way-points. The segment list can be included by the original source of the packet or by an intermediate node. When the segment list is inserted by an intermediate node, it can be removed by another node along the path of the packet, supporting the concept of \textit{tunneling} through an \textit{SR domain} from an \textit{ingress} node to an \textit{egress} node. 

The implementation of the Segment Routing Architecture requires a data plane which is able to carry the segment lists in the packet headers and to properly process them. Control plane operations complement the data plane functionality, allowing to allocate segments (i.e. associate a segment identifier to a specific instruction in a node) and to distribute the segment identifiers within an SR domain.

As for the data plane, two instantiations of the SR Architecture have been designed and implemented, SR over MPLS (SR-MPLS) and SR over IPv6 (SRv6). With SR-MPLS, no change to the MPLS forwarding plane is needed \cite{id-segment-routing-mpls}. SRv6 is based on a new type of IPv6 routing header called SR Header (SRH) \cite{ietf-6man-segment-routing-header}. Temporally, SR-MPLS has been the first instantiation of the SR Architecture, while the recent interest and developments are focusing on SRv6. In particular, the IPv6 data plane for SR is being extended to support the so-called SRv6 Network Programming Model \cite{id-srv6-network-prog}. According to this model, Segment Routing functions can be combined to achieve an end-to-end (or edge-to-edge) networking objective that can be arbitrarily complex. This is appealing for implementing complex services like Service Function Chaining. SRv6 can be used as an \textit{overlay} tunneling mechanism directly exposed and used by servers (similar to VXLAN tunneling) and as an \textit{underlay} transport mechanism in network backbones (supporting Traffic Engineering and Resilience services). In this vision, SRv6 can simplify network architectures avoiding the use of different protocol layers. 

As for the SR Control Plane operations, they can be based on a distributed, centralized or hybrid approach. In the distributed approach, the routing protocols are used to signal the allocation of segments and the nodes take independent decisions to associate packets to the segment lists. In the centralized approach, an SR controller allocates the segments, takes the decision on which packets needs to be associated to which segment lists and configures the nodes accordingly. Very often, an hybrid approach which consists in the combination of distributed and centralized approach is used (see for example \cite{ventre2018sdn}). 

The goal of this paper is to provide a comprehensive survey on the Segment Routing technology, including all the achieved results and the ongoing work. Hereafter (section \ref{sec:sr-evolution}, we start with the historical context on the development of Segment Routing. In section \ref{sec:arch}, we provide an introduction to the main concepts of the Segment Routing architecture. We consider both the SR-MPLS and the SRv6 data planes, but we focus more deeply on SRv6 which is currently rising a lot of interest. In section~\ref{sec:standard}, we provide a classification and a discussion of the standardization efforts. We provide a comprehensive review of the research activities in section~\ref{sec:research}, covering \numTotalPapers scientific papers. The most relevant implementation results and the status of ongoing deployments are reported in section~\ref{sec:tools}. We highlight future research directions and open issues in section \ref{sec:future}.  

\subsection{Segment Routing roots and evolution}
\label{sec:sr-evolution}

The Source Routing approach consists in including the route of the packet as a list of hops in the packet header and it has two variants. \textit{Strict} Source Routing requires to specify the full sequence of hops from the source to the destination. \textit{Loose} source routing consists in specifying a list of nodes that represents \textit{way points} to be crossed (in their order) before reaching the destination. These two variants of Source Routing has been considered among the possible solutions for packet routing and forwarding since the early phases of design of the packet switching technologies. In particular, they have been considered in the original definition of IPv4 protocol in the late 1970's. RFC 791 \cite{rfc791}, which defined IPv4 in 1981, included the \textit{Strict Source and Record Route} (SSRR) and the \textit{Loose Source and Record Route} (LSRR) options in the IPv4 packet header. These options have been rarely used in IPv4 networks, also due to security issues. Packets carrying the SSRR or LSRR options are typically filtered (dropped) by IPv4 routers in the Internet.

Segment Routing follows the \textit{loose} variant of Source Routing, using the same approach of IPv4 Loose Source Routing, but it is specifically based on MPLS or IPv6 data planes. The research and standardization activities on Segment Routing originated in the late 2000's, mainly with the goal of overcoming some scalability issues and limitations \cite{rfc5439} that had been identified in the traffic engineered Multi Protocol Label Switching (MPLS-TE) solutions used for IP backbones. In particular it was observed that MPLS-TE requires explicit state to be maintained at all hops along an MPLS path and this may lead to scalability problems in the control-plane and in the data-plane. Moreover, the per-connection traffic steering model of MPLS-TE does not easily exploit the load balancing offered by Equal Cost MultiPath (ECMP) routing in plain IP networks. On the other hand, Segment Routing can steer traffic flows along traffic engineered paths with no per-flow state in the nodes along the path and exploiting ECMP routing within each segment. 

In the early 2010's, the IETF started the ``Source Packet Routing in Networking'' Working Group (SPRING WG) to deal with Segment Routing. The activity of the SPRING WG has included the identification of Use Cases and Requirements for Segment Routing (for example, \cite{rfc7855}, \cite{rfc8355} and \cite{rfc8354} have become IETF RFCs). Recently (July 2018), the SPRING WG has issued the ``Segment Routing Architecture'' document (RFC 8402 \cite{rfc8402}), while several other documents are still under discussion by the WG, as it will be analyzed later in this paper.

Looking at the scientific bibliography, the seminal paper on the Segment Routing Architecture is \cite{filsfils2015segment}. Published in 2014, it provides an overview of the motivations for SR, describes a set of important use cases and illustrates the architecture. The basic concepts proposed in \cite{filsfils2015segment} have been elaborated and refined in the RFC 8402 \cite{rfc8402} standardized by IETF. 

Currently, Segment Routing is receiving a lot of interest from operators for its applications in different types of networks (transport backbones, access networks, datacenters, 5G networks). The MPLS based data plane (SR-MPLS) relies on the well established MPLS technology. SR-MPLS can be seen as an improvement and a simplification of the traditional MPLS control plane, so it is beneficial to operators with an already deployed MPLS backbone. The IPv6 based data plane (SRv6) is gaining traction as it offers the possibility to combine overlay and underlay networking services and features only using the IPv6 technology. The SRv6 \textit{network programming model} offers unprecedented flexibility is designing and operating network services, so SRv6 is an attractive choice for operators that are deploying new networks or planning the evolution of their networking architectures.

We conclude this short historical review by noting that a very large number of patents have been registered related to Segment Routing, as it is possible to verify with a cursory on-line search. We are not analyzing these patents in our survey, but they demonstrate the high interest of vendors and service providers in the SR technology.


