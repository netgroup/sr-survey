\section{Introduction}
\label{sec:intro}

\IEEEPARstart{S}egment Routing (SR) is based on the loose Source Routing concept. A node can include an ordered list of instructions in the packet headers. These instructions steer the forwarding and the processing of the packet along its path in the network. The single instructions are called \textit{segments}, a sequence of instructions can be referred to as a \textit{segment list} or as an \textit{SR Policy} . Each segment can enforce a topological requirement (e.g. pass through a node or an interface) or a service requirement (e.g. execute an operation on the packet). The term \textit{segment} refers to the fact that a network path towards a destination can be split in segments by adding intermediate way-points. The segment list can be included by the original source of the packet or by an intermediate node. When the segment list is inserted by an intermediate node, it can be removed by another node along the path of the packet, supporting the concept of \textit{tunneling} through an \textit{SR domain} from an \textit{ingress} node to an \textit{egress} node. 

The research and standardization activities on Segment Routing originated in the late 2000's, mainly with the goal of overcoming some scalability issues and limitations \cite{rfc5439} that had been identified in the traffic engineered Multi Protocol Label Switching (MPLS-TE) solutions used for IP backbones. In particular it was observed that MPLS-TE requires explicit state to be maintained at all hops along an MPLS path and this may lead to scalability problems in the control-plane and in the data-plane. Moreover, the per-connection traffic steering model of MPLS-TE does not easily exploit the load balancing offered by Equal Cost MultiPath (ECMP) routing in plain IP networks. On the other hand, Segment Routing can steer traffic flows along traffic engineered paths with no per-flow state in the nodes along the path and exploiting ECMP routing within each segment. 

In the early 2010's, the IETF started the ``Source Packet Routing in Networking'' Working Group (SPRING WG) to deal with Segment Routing. The activity of the SPRING WG has included the identification of Use Cases and Requirements for Segment Routing (for example, \cite{rfc7855}, \cite{rfc8355} and \cite{rfc8354} have become IETF RFCs). Recently, the WG has issued the ``Segment Routing Architecture'' document (RFC 8402 \cite{rfc8402}), while several other documents are still under discussion by the WG, as it will be analyzed later in this paper.

The implementation of the Segment Routing Architecture requires a Data Plane which is able to carry the segment lists in the packet headers and to properly process them. Control Plane operations complement the Data Plane functionality, allowing to allocate segments (i.e. associate a segment identifier to a specific instruction in a node) and to distribute the segment identifiers within an SR domain.

As for the Data Plane, two instantiations of the SR Architecture have been designed and implemented, SR over MPLS (SR-MPLS) and SR over IPv6 (SRv6). With SR-MPLS, no change to the MPLS forwarding plane is needed \cite{id-segment-routing-mpls}. SRv6 is based on a new type of IPv6 routing header called SR Header (SRH) \cite{draft-srh}. Temporally, SR-MPLS has been the first instantiation of the SR Architecture, while the recent interest and developments are focusing on SRv6. In particular, the IPv6 SR dataplane is being extended to support the so-called SRv6 Network Programming Model \cite{id-srv6-network-prog}. According to this model, segment routing functions can be combined to achieve an end-to-end (or edge-to-edge) networking objective that can be arbitrarily complex. 

As for the SR Control Plane operations, they can be based on a distributed, centralized or hybrid approach. In the distributed approach, the routing protocols are used to signal the allocation of segments and the nodes take independent decisions to associate packets to the segment lists. In the centralized approach, an SR controller allocates the segments, takes the decision on which packets needs to be associated to which segment lists and configures the nodes accordingly. Very often, an hybrid approach which consists in the combination of distributed and centralized approach is used. 

The goal of this paper is to provide a comprehensive survey on the Segment Routing technology, including all the achieved results and the ongoing work. We will consider the standardization efforts (section IV), the research activities (section III), the implementation results (section V) and ongoing deployments (section VI). In addition, in section II we provide a short introduction to the main concepts of the Segment Routing architecture, while we highlight future research directions and open issues in section VII.  

To the best of our knowledge, there is only another survey about Segment Routing, namely \cite{abdullah2018segment}. With respect to \cite{abdullah2018segment}, our survey is more complete in the analysis of the scientific literature, covering more than 75 papers. Moreover, it provides a classification and discussion of standardization activities, and considers the status of implementations \extended{and deployments of Segment Routing} (the analysis of standardization efforts and implementation results are both missing in \cite{abdullah2018segment}). 